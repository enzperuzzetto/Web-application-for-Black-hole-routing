\section{Cahier des Charges}
Après avoir analysé les outils existants et les besoins du client, on s'est rendu compte que le client aura besoin d'une application Web de type RESTful pour pouvoir interagir avec ExaBGP.
\newline
Par conséquent, notre application Web s'appuiera sur le framework Meteor JS. 

\subsection{Besoins fonctionnels}

\subsubsection{Administrateur}
\begin{enumerate}

\item S'identifier à l'aide d'un identifiant et un mot de passe
	\begin{itemize}
		\item les informations de connexion doivent être stockées dans la base de donnée
	\end{itemize}

\item Vérifier le bon fonctionnement ExaBGP :
	\begin{itemize}
        \item Envoyer un message d'avertissement(popup)quand ExaBGP est en panne ou ne tourne pas.
	\end{itemize}

\item Annoncer une route : 
	\begin{itemize}
    	\item Envoyer une requête HTTP composé du fichier JSON au serveur ExaBGP, pour exécuter la commande 'announce route'.
		\item Annoncer un réseau ou une adresse IP à tous les BGP 'peers' afin qu'ils prennent en compte la configuration de la route entrée par l'administrateur réseau.
        \item Attribuer une communauté à une adresse IP ou à un réseau lors de l'ajout.
        \item La route sera stockée dans la base de données.
        \item En cas de succès, un message s’apparaît à l'utilisateur pour confirmer l'ajout. Dans le cas échéant, un message d'erreur est affiché à l'écran contenant le code d'erreur. 
	\end{itemize}
    
    \item Pouvoir supprimer une route :
	\begin{itemize}
		\item L'administrateur réseaux pourrait supprimer une route qui a été déjà annoncée.
        \item La route sera donc supprimée de la base de données.
        \item En cas de succès, un message s’apparaît à l'utilisateur pour confirmer la suppression. Dans le cas échéant, un message d'erreur est affiché à l'écran contenant le code d'erreur. 

	\end{itemize}
    
\item Dévier une route vers un trou noir : 
	\begin{itemize}
		\item La route sera déviée vers le trou noir selon son adresse IP source, sa destination ou bien la communauté à laquelle elle appartient.
        \item Le Next-Hop de la route sera l'interface null0.
        \item En cas de succès, un message s’apparaît à l'utilisateur pour confirmer que la route n'est plus accessible selon le préfixe définit(Source IP ou destination IP ou bien la communauté). Dans le cas échéant, un message d'erreur est affiché à l'écran contenant le code d'erreur. 

	\end{itemize}

\item Relancer ExaBGP en cas de panne :
	\begin{itemize}
		\item Un bouton permettant d'envoyer une requête HTTP au serveur ExaBGP.
        \item La requête consiste à exécuter la commande 'reload' de l'API de ExaBGP.
        \item La commande 'reload' relance la configuration d'ExaBGP sur le serveur.
        \item En cas de succès, un message de confirmation s’apparaît pour l’utilisateur. Dans le cas échéant, un message d'erreur s’apparaît portant le code d'erreur.
	\end{itemize}

\item Exécuter les différentes commandes de l'API de ExaBGP
	\begin{itemize}
        \item L'application web devrait pouvoir exécuter la totalité des commandes de L'API ExaBGP.
        \item Les commandes à exécuter seront envoyées au serveur ExaBGP en format fichier JSON.
        \item En cas d'erreur, l'application web devrait récupérer le code d'erreur de l'exécution provenant du serveur ExaBGP.
	\end{itemize}	
\end{enumerate}

\subsubsection{Administrateur et client}
\begin{enumerate}

\item Lister les routes :
	\begin{itemize}
    	\item Lorsque qu'une route est modifié, ses informations sont mises à jour dans la base de donnée
		\item L'utilisateur peut voir toutes les routes qui existent dans la base de données.
		\item Le résultat sera découpé en plusieurs pages web pour facilité la lisibilité.
	\end{itemize}   
    
\item Rechercher des routes selon leurs préfixes, (adresse IP, communauté, destination) :
	\begin{itemize}
		\item Une page web dans l'application dédiée à effectuer la recherche des routes selon leurs préfixes.
	\end{itemize}
    
\end{enumerate} 



\subsection{Besoins non fonctionnels}

%(2) mettre les scripts python et JS : une contrainte
\begin{itemize}
\item Certains packages de meteor.js
	\begin{itemize}
		\item twbs:bootstrap
        \item ian:accounts-ui-bootstrap-3
        \item iron:router
	\end{itemize}
\item Synchronisation du serveur web avec ExaBGP
\item Interface différente pour l'admin et l'utilisateur anonyme
\item Ecriture des scripts en Python et Javascript (imposé par l'utilisation de ExaBGP et Meteor)
\end{itemize}

\subsubsection{Organisation}
En effet, pour qu'on puisse commencer la phase de développement, on a besoin de virtualiser la topologie complète de réseaux et donc d'installer l'environnement technique sur nos machines (Dynamips, Nemu, et ExaBGP).
\\
\\
Concernant l'ordre de priorité, nous avons étudié les besoins du client et on en avait déduit l'ordre suivant : 
\begin{enumerate}
\item Installer l'environnement technique dans le but d'avoir une topologie virtuelle.
\item Réaliser les tâches qui concerne l'administrateur en partant de la tâche '1' jusqu'à la tâche '5'.
\item Réaliser les tâches qui concerne l'administrateur et le client dans l'ordre indiqué.
\item Enfin, réaliser la tâche '6' et '7' concernant l'administrateur.
\end{enumerate}